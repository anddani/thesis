I denna studie undersöks prestandaskillnader mellan Java-kod kompilerad av Android Runtime och C++-kod kompilerad av Clang. För experimenten som mätte skillnaderna användes en Fast Fourier Transform (FFT) för att understryka vilka användningsområden som kräver hög prestanda på en \hl{mobil enhet / mobilenhet / smart telefon / smartphone}. Olika påverkande aspekter vid användningen av en FFT undersöktes. Ett test undersökte hur mycket påverkan Java Native Interface (JNI) hade på ett program i helhet. Resultaten från dessa tester visade att påverkan var mycket liten för FFT-storlekar större än 64. Ett annat test undersökte prestandaskillnader mellan program översatta från Java till C++. Slutsatsen kring dessa tester var att av de översatta algoritmerna var Columbia Iterative FFT den som presterade bäst, både i Java och i C++. \hl{Vektorisering} visade sig vara en effektiv optimeringsteknik för maskinkodskompilerad kod skriven i C++. Slutligen utfördes tester som undersökte prestandaskillnader mellan flyttalsprecision för datatyperna \texttt{float} och \texttt{double}. \texttt{float} kunde förbättra prestandan genom att på ett effektivt sätt utnyttja processorns cache.
