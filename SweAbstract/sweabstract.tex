I denna studie undersöktes prestandaskillnader mellan Java-kod kompilerad av Android Runtime och C++-kod kompilerad av Clang på Android. En Fast Fourier Transform (FFT) användes under experimenten för att visa vilka användningsområden som kräver hög prestanda på en mobil enhet. Olika påverkande aspekter vid användningen av en FFT undersöktes. Ett test undersökte hur mycket påverkan Java Native Interface (JNI) hade på ett program i helhet. Resultaten från dessa tester visade att påverkan inte var signifikant för FFT-storlekar större än 64. Ett annat test undersökte prestandaskillnader mellan FFT-algoritmer översatta från Java till C++. Slutsatsen kring dessa tester var att av de översatta algoritmerna var Columbia Iterative FFT den som presterade bäst, både i Java och i C++. Vektorisering visade sig vara en effektiv optimeringsteknik för arkitekturspecifik kod skriven i C++. Slutligen utfördes tester som undersökte prestandaskillnader mellan flyttalsprecision för datatyperna \texttt{float} och \texttt{double}. \texttt{float} kunde förbättra prestandan genom att på ett effektivt sätt utnyttja processorns cache.
