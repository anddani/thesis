This thesis investigates the performance differences between Java code compiled by Android Runtime and C++ code by Clang. For testing the differences, the Fast Fourier Transform (FFT) algorithm was chosen to demonstrate examples of when it is relevant to have high performance computing on a mobile device. Different aspects that could affect the execution time of a program were examined. One test measured the overhead related to the Java Native Interface (JNI) was tested. The results showed that the overhead was insignificant for common sizes of the FFT. Another test compared equivalent implementations between Java and native code. The conclusion of this test was that the Columbia Iterative FFT had consistent results between programming languages.

% Another test The experiments performed in this thesis showed that Java code compiled by Android Runtime perform similar to equivalent C++ code given that 
