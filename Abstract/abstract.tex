This thesis investigates the performance differences between Java code compiled by Android Runtime and C++ code compiled by Clang. For testing the differences, the Fast Fourier Transform (FFT) algorithm was chosen to demonstrate examples of when it is relevant to have high performance computing on a mobile device. Different aspects that could affect the execution time of a program were examined. One test measured the overhead related to the Java Native Interface (JNI). The results showed that the overhead was insignificant for FFT sizes larger than 64. Another test compared equivalent implementations between Java and native code. The conclusion drawn from this test was that, of the converted algorithms, Columbia Iterative FFT performed the best in both Java and C++. Vectorization proved to be an efficient optimization alternative for native development. Finally, tests examining the effect of using single-point precision (\texttt{float}) versus double-point precision (\texttt{double}) data types were covered. Choosing \texttt{float} could improve performance by an efficient use of cache.

\textbf{Keywords:} \emph{Android, NDK, Dalvik, Java, C++, Native, Android Runtime, Clang, Java Native Interface, Digital Signal Processing, Fast Fourier Transform, Vectorization, NEON, Performance Evaluation}
