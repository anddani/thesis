%%%%%%%%%%%%%%%%%%%%%%%%%%%%%%%%%%%%%%%%%
% Beamer Presentation
% LaTeX Template
% Version 1.0 (10/11/12)
%
% This template has been downloaded from:
% http://www.LaTeXTemplates.com
%
% License:
% CC BY-NC-SA 3.0 (http://creativecommons.org/licenses/by-nc-sa/3.0/)
%
%%%%%%%%%%%%%%%%%%%%%%%%%%%%%%%%%%%%%%%%%
\documentclass{beamer}
\usepackage[utf8]{inputenc}

\mode<presentation> {%

% The Beamer class comes with a number of default slide themes
% which change the colors and layouts of slides. Below this is a list
% of all the themes, uncomment each in turn to see what they look like.

% \usetheme{default}
% \usetheme{AnnArbor}
% \usetheme{Antibes}
% \usetheme{Bergen}
% \usetheme{Berkeley}
% \usetheme{Berlin}
\usetheme{Boadilla}
% \usetheme{CambridgeUS}
% \usetheme{Copenhagen}
% \usetheme{Darmstadt}
% \usetheme{Dresden}
% \usetheme{Frankfurt}
% \usetheme{Goettingen}
% \usetheme{Hannover}
% \usetheme{Ilmenau}
% \usetheme{JuanLesPins}
% \usetheme{Luebeck}
% \usetheme{Madrid}
% \usetheme{Malmoe}
% \usetheme{Marburg}
% \usetheme{Montpellier}
% \usetheme{PaloAlto}
% \usetheme{Pittsburgh}
% \usetheme{Rochester}
% \usetheme{Singapore}
% \usetheme{Szeged}
% \usetheme{Warsaw}

% As well as themes, the Beamer class has a number of color themes
% for any slide theme. Uncomment each of these in turn to see how it
% changes the colors of your current slide theme.

%\usecolortheme{albatross}
%\usecolortheme{beaver}
%\usecolortheme{beetle}
%\usecolortheme{crane}
%\usecolortheme{dolphin}
%\usecolortheme{dove}
%\usecolortheme{fly}
%\usecolortheme{lily}
%\usecolortheme{orchid}
%\usecolortheme{rose}
%\usecolortheme{seagull}
%\usecolortheme{seahorse}
%\usecolortheme{whale}
%\usecolortheme{wolverine}

%\setbeamertemplate{footline} % To remove the footer line in all slides uncomment this line
%\setbeamertemplate{footline}[page number] % To replace the footer line in all slides with a simple slide count uncomment this line

%\setbeamertemplate{navigation symbols}{} % To remove the navigation symbols from the bottom of all slides uncomment this line
}

\usepackage{graphicx} % Allows including images
\usepackage{booktabs} % Allows the use of \toprule, \midrule and \bottomrule in tables
\usepackage{lmodern}%


\newcommand{\thetitle}{Comparing Android Runtime with native:\\Fast Fourier Transform on Android}
\title[Thesis presentation]{\thetitle} % The short title appears at the bottom of every slide, the full title is only on the title page

\author{André Danielsson} % Your name
\institute[KTH] % Your institution as it will appear on the bottom of every slide, may be shorthand to save space
{%
\textit{anddani@kth.se}\\ % Your email address
\medskip
Royal Institute of Technology\\
Computer Science and Communication\\ % Your institution for the title page
}
\date{\today} % Date, can be changed to a custom date

\AtBeginSection[]
{%
    \begin{scriptsize}
        \begin{frame}<beamer>{Outline}
            \tableofcontents[currentsection]
        \end{frame}
    \end{scriptsize}
}

\begin{document}

%------------------------------------------------
%	PRESENTATION SLIDES
%------------------------------------------------

% 1.  Present the subject. Describe the work in 1-2 sentences (Intro)
% 2.  Outline
% 3.  Why is this important?
%     Where is it used?
%     Who can benefit from it?
% 4.  Research question
% 5.  Present the necessary knowledge.
%     ART
%     JNI
%     DFT/FFT
%     Previous studies
% 6.  Method
%     Present a list of tests (JNI, libs, opts, float vs double, gc)
%     What is measured?
%     How is it measured?
%     Why is it relevant?
% 7.  Implementation
%     Discuss why the algorithms were chosen
%     NEON
% 8.  Results/Discussion
% 9.  Conclusions
% 10. Questions

%%%------------------------------------------------
%%% Title page, present the work
%%%------------------------------------------------
\begin{frame}
\titlepage%
\end{frame}

%%%------------------------------------------------
%%% Presentation outline
%%%------------------------------------------------
\begin{frame}{Outline}
    \begin{scriptsize}
        \tableofcontents
    \end{scriptsize}
\end{frame}

%%%------------------------------------------------
%%% Introduction
%%%------------------------------------------------
\section{Introduction}
%------------------------------------------------
% Purpose of work
%------------------------------------------------
\subsection{Purpose of Work}
\begin{frame}{Purpose of Work}
    \begin{itemize}
        \item<1-> Why is this important?
        \item<2-> Where is it used?
        \item<3-> Who can benefit from it?
    \end{itemize}
\end{frame}

%------------------------------------------------
% Research Question
%------------------------------------------------
\subsection{Research Question}
\begin{frame}{Research Question}
\centering
\emph{Is there a significant performance difference between implementations of a Fast Fourier Transform (FFT) in native code, compiled by Clang, and Dalvik bytecode, compiled by Android Runtime, on Android?}
\end{frame}

%%%------------------------------------------------
%%% Background
%%%------------------------------------------------
\section{Background}

%------------------------------------------------
% ART
%------------------------------------------------
\subsection{Android Platform}
\begin{frame}{Android Platform}
\end{frame}

%------------------------------------------------
% JNI
%------------------------------------------------
\subsection{Java Native Interface (JNI)}
\begin{frame}{Java Native Interface (JNI)}
\end{frame}

%------------------------------------------------
% DFT
%------------------------------------------------
\subsection{Discrete Fourier Transform (DFT)}
\begin{frame}{Discrete Fourier Transform (DFT)}
\end{frame}

%------------------------------------------------
% FFT
%------------------------------------------------
\subsection{Fast Fourier Transform (FFT)}
\begin{frame}{Fast Fourier Transform (FFT)}
\end{frame}

%------------------------------------------------
% Related Work
%------------------------------------------------
\subsection{Related Work}
\begin{frame}{Related Work}
\end{frame}

%%%------------------------------------------------
%%% Method
%%%------------------------------------------------
\section{Method}

%------------------------------------------------
% Experiments
%------------------------------------------------
\subsection{Experiments}
\begin{frame}{Experiments}
\end{frame}

%------------------------------------------------
% Measurements
%------------------------------------------------
\subsection{Measurements}
\begin{frame}{Measurements}
\end{frame}

%------------------------------------------------
% Implementation
%------------------------------------------------
\subsection{Implementation}
\begin{frame}{Implementation}
\end{frame}

%%%------------------------------------------------
%%% Results and Discussion
%%%------------------------------------------------
\section{Results and Discussion}
\subsection{JNI}
\begin{frame}{JNI}
\end{frame}
\subsection{Libraries}
\begin{frame}{Libraries}
\end{frame}
\subsection{NEON}
\begin{frame}{NEON}
\end{frame}
\subsection{\texttt{float} vs \texttt{double}}
\begin{frame}{\texttt{float} vs \texttt{double}}
\end{frame}
\subsection{Garbage Collection}
\begin{frame}{Garbage Collection}
\end{frame}

%%%------------------------------------------------
%%% Conclusions
%%%------------------------------------------------
\section{Conclusions}
\begin{frame}{Conclusions}
    \begin{block}{Conclusion 1}
        
    \end{block}
    \begin{block}{Conclusion 2}

    \end{block}
    \begin{block}{Conclusion 3}

    \end{block}
\end{frame}

% %------------------------------------------------
%
% \begin{frame}
% \frametitle{Native Development Kit}
% \begin{columns}[c] % The "c" option specifies centered vertical alignment while the "t" option is used for top vertical alignment
%
% \column{.45\textwidth} % Left column and width
% \textbf{Heading}
% \begin{enumerate}
% \item Statement
% \item Explanation
% \item Example
% \end{enumerate}
%
% \column{.5\textwidth} % Right column and width
% Lorem ipsum dolor sit amet, consectetur adipiscing elit. Integer lectus nisl, ultricies in feugiat rutrum, porttitor sit amet augue. Aliquam ut tortor mauris. Sed volutpat ante purus, quis accumsan dolor.
%
% \end{columns}
% \end{frame}

\begin{frame}
\Huge{\centerline{Questions?}}
\end{frame}

\end{document}
