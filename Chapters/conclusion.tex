\textit{The conclusions drawn from the discussion will be presented in this chapter as well as an answer to the scientific question of this thesis. Propositions of future work will also be included.}

%%%
% JNI Conclusion
JNI overhead is very small, around 1-10 \textmu s for a Qualcomm MSM8994 Snapdragon 810 chip. If a program were to call native code in a loop, it could add up to some overhead. It is only for very small block sizes around $2^4 - 2^6$ where this overhead becomes a substantial part of the total execution time. For larger and more common block sizes, the overhead from JNI is not significant.

%%%
% FFT Conclusion
One of the conclusions to draw from the results is that when choosing Java, the best algorithm was the Columbia Iterative. It was the easiest to convert to C++ because it did not include any Java specific constructs (except for method definitions). It was also the fastest between it, Princeton Iterative and Princeton Recursive for both Java and C++. In Columbia Iterative, after the initial setup of calculating the trigonometric tables and allocating space for the input/output array, no memory allocation is done during the actual transform.

%%%
% Garbage Collection Conclusion
To prevent garbage collection pauses throughout the program, it is necessary to pre-allocate data structures that should be used in a function that is called multiple times instead of allocating new memory for each call. By doing this, it reduces the work for the garbage collector. Less allocated memory will go out of scope, making garbage collection pauses shorter and less frequent.

% Optimization Conclusion
Vectorization was a very efficient optimization which performed much better than their corresponding algorithms in C++. Native optimization is architecture dependent. This means that an application will only work on a subset of Android devices running some architectures the program is compiled for. Further, some optimizations are only possible on a few architectures. This makes some optimizations better than other. An example of a more portable optimization is parallelization.

%%%
% Float and Doubles Conclusion
Operations between elements of the \texttt{float} data structure resulted in faster execution time than between \texttt{double}s on the Nexus 6P. It is therefore important to know if \texttt{float} gives sufficient precision for the given application. If high precision is important, you should use a \texttt{double} as primary data type. \texttt{float} is useful for architectures where it is faster to compute between floats or when memory efficiency is important.

%%%
% Answer research question
To answer the research question, based on the results from the experiments, we can see that, of the chosen algorithms, there is not a significant performance difference between the fastest FFT library for Java with its corresponding implementation in native code. Optimization in native code does make native code significantly faster than the fastest Java implementation. This does increase the complexity of the code and decrease the compatibility between devices.

%%%
% Future work
One interesting aspect that could be examined for future work is parallelization. This optimization has more compatibility than the optimization in NEON because it can be used in Java. There are different libraries to compare a possible implementation with such as JTransforms~\cite{jtransforms:benchmark}.
