\textit{The discussion chapter covers how the JNI affects performance, how efficient smaller FFT libraries are, why the optimization gave the results found in Chapter~\ref{ch:results} - Results and the efficiency difference between floats and doubles.}

\section{JNI Overhead}
The test results from the JNI tests showed that the overhead is small relative to the computation of the FFT algorithm. As long as it is being run once per calculation, it will not affect the performance significantly. If the JNI is called in a loop when it might not be necessary, the overhead can add up and become a larger part of the total execution time. Another thing to note is that the execution time stay within about 10 \textmu s.

The confidence intervals overlap for many of the values, meaning we cannot say whether one input yields a faster execution time than the other. Some larger block sizes has lower execution time than smaller block sizes and some grow for larger input. It is then reasonable to assume that nothing is done to the arrays when they are passed to the JNI, only pointers are copied. The \texttt{GetPrimitiveArrayCritical} and \texttt{ReleasePrimitiveArrayCri\-tical} seem to introduce overhead when used on larger arrays. % <= CITE THIS ... as described in

Regarding the spike in mean for some JNI results, this can be a cause of one large execution time skewing the results. This is actually the case for the \textbf{Convert} tests with block size 1024 as seen in Figure~\ref{fig:raw:jni:convert:1024} found in Appendix B. A reason for this could be that the garbage collector began executing during the timing of the test.

\section{Simplicity and Efficiency}

The slowest algorithm was the Java Princeton Recursive. The reason for this is because it executes many method calls. For each method call, registers must be saved by pushing them onto a stack, function arguments must them be moved to appropriate registers and a jump instruction to the new code must be executed. When the function has finished executing, the registers are popped from the stack. This causes a lot of overhead.

Additionally, each call creates new \texttt{Complex} arrays when splitting up the array in odd and even indices. Lastly, when combining the arrays, new \texttt{Complex} objects are created each time an operation is done between two complex numbers. The reason for the is because the \texttt{Complex} class creates immutable objects. This slows down the process and increases memory consumption, increasing work for the garbage collector.

The C++ version of the Princeton Recursive algorithm is faster than the Java version. One big difference is that the \texttt{std::complex} type used in C++ does not create new instances each time is is being operated with. Instead, they are placed on the stack. This lowers the number of calls requesting more memory from the heap. Depending on the situation for the program, there is a risk that the program must ask the system for more memory, slowing down the allocation process.

Additionally, this will increase the work for the garbage collector, increasing the risk of it being triggered during the tests. To prevent the number of allocations you are doing inside a repeated process you commonly reuse allocated memory. This is done by pre-allocating the necessary arrays or other data structures and overwrite the results for each call. Avoiding calls to the \texttt{new} keyword in a method that is called multiple times can increase time and memory efficiency.

Of the algorithms tested in the FFT library tests, KISS FFT was the fastest. It is more optimized than the \enquote{basic} implementations found in the Columbia and Princeton algorithms. In C++, Princeton Iterative and Columbia Iterative were the fastest and in Java, Columbia Iterative was the fastest. The reason for Princeton Iterative being faster in C++ is, as for Java Princeton Recursive, because it uses the \texttt{Complex} class to represent complex numbers. Because Java Columbia Iterative used double arrays to represent real and imaginary numbers, no new objects needed to be created.

The results from Chapter~\ref{ch:gc} - Garbage Collection show that the Java Princeton algorithms caused the GC to run. This behaviour was as expected because they called new during their execution. Java Columbia did not cause any garbage collection and neither did any of the C++ algorithms.

One thing that is clear is that the Columbia Iterative algorithm is the best one to choose from of the Java versions. It performs better than both Princeton Iterative and Princeton Recursive. It also allows simple modifications such as changing between using \texttt{float}s or  \texttt{double}s to represent the data.

% The reason Princeton Iterative and Princeton Recursive is slower in Java than in C++ is because they operate with \texttt{Complex} elements. Each time two \texttt{Complex} numbers are added, multiplied or subtracted, a new \texttt{Complex} object is created.

As we have seen in the tests, choosing an iterative implementation is preferable and choosing the correctly implemented iterative implementation is also important. It is possible to compare implementations by setting up small tests. This is a small time investment that ensures that you get the performance you need in a program. It is also possible to read through the implementation, looking for places where it allocates memory and edit it so that it uses an already allocated array. This reduces the number of calls for more memory which lessens the chance of a garbage collect.

Comparing the Java and C++ implementations of the Columbia Iterative FFT, for \emph{small}-\emph{large} block sizes, the Java version is almost as fast as the C++ version. For the block sizes added in the \emph{extra large} group, the C++ version perform better. If you were to choose between the C++ and the Java variant of Columbia, it is better to choose the Java version if the performance requirement allows it. Avoiding having to implement a JNI bridge between Java and native is more preferable than the small increase in performance.

Another argument why it is sensible to choose the Java version is that is has enough performance for processing to allow graphical rendering in 60 \gls{fps}. Let's say we have an application that visualizes sound with a sampling frequency of 44100 Hz and updates the screen at 60 frames per second. It has 16.66 ms to do all the processing needed between the screen updates. If we want to have double precision numbers and use Java it is possible to do an FFT with a size of \textbf{32768}.

% => More common to use sliding window, minimize latency <=

This size is more than needed and would require $32768/44100\approx 743 $ ms to sample. The fastest Java algorithm for double point precision ran in $12$ ms for this block size. As we can see, in relation, the time it takes to execute an FFT is only a small fraction of the time it takes to gather the same number of samples. This is important because it allows larger FFT sizes from the sampled audio, increasing the frequency resolution in the results, while still being sufficiently fast.

For smaller block sizes (\emph{medium} and \emph{large}), common when sampling low frequencies such as speech and acceleration data, it is relevant to have fast transforming for these sizes. Comparing C++ and Java we see that Java is not that much slower, at best, than C++. This gives an incentive to use Java, allowing more consistent code and does not add the complexity of JNI.

When the block sizes are large, a bigger difference is found between the algorithms. One reason for this is the impact the garbage collection has when triggered during the timing of an algorithm. The pauses can range between, depending on the work done in the garbage collect, 0.468-23.760 ms.

KISS FFT was the fastest of all the native implementations. This was because of the optimizations done to improve its performance. Using multiple radix implementations of the FFT, it is possible to get more performance at the cost of higher code complexity. It shows the potential in doing small optimizations that does not rely on multi-core or other architecture dependent optimization techniques.

How memory is used differs between algorithms. As seen in Chapter~\ref{ch:gc}, only four tests caused the garbage collector to be run. Java Princeton Iterative and Recursive with both \texttt{float} and \texttt{double} as primary data types were the algorithms causing this. This is because they create new arrays each time they are called. The garbage collector needed to remove each array that was allocated each call, thus increasing the total work the garbage collector must achieve. It is also possible to see that for the \texttt{double} data type, garbage collection are caused earlier than for \texttt{float} during the tests.

\section{Vectorization as Optimization}

As the results show, vectorization was an improvement in regards to efficiency. This was as expected because it is possible to process more elements for each instruction. The optimizations implicitly implemented loop unrolling because more operations were covered in the body of the loop, resulting in less jump instructions, leading to less instructions in total.

The Recursive NEON implementation did not perform much worse than the Iterative version. The difference between the implementations were of a smaller factor than between Princeton Iterative and Recursive. One reason for this could be that because the twiddle factors are moved so that they are closer in cache in a two dimensional array for the recursive version. They are placed in order of access to reduce the number of cache misses. This could be the reason for the performance increase.

% IMPRTNT => MORE PROS?? <= IMPRTNT 

One disadvantage of using NEON intrinsics is the fact that you cannot run it on all CPU architectures. Not all Android devices runs on ARM, and not all ARMv7 supports NEON \cite{arm:neon}. This makes it less compatible if the program depends on its optimizations. One example of an optimization that is architecture independent and works on all multi-core CPUs is parallelization. Another disadvantage of NEON is that it makes a program harder to maintain. It introduces complexity to the program in addition to JNI. 

A possible implementation could be optimized by fitting all the components in cache. This reduces the overhead of fetching data from memory which would slow down the overall performance. If it is not possible for the data to fit in cache, you can rewrite the algorithm such that the elements are stored close by in order of access. This will result in less cache invalidation and less direct memory reads.

When needing to do a lot of processing with the data in frequency domain, a fast FFT will help in lowering the total execution time. If the data is also transformed back into time domain, for example when you want to filter out noise in an audio signal and play it back. Another reason you would want a fast FFT is when the computation time for some validation need to be as fast as possible. An example of this is voice recognition where we want answers in reasonable time. Another example could be to do image recognition on fingerprints for fast biometric authentication.

\section{Floats and Doubles}
One reason for using the \texttt{float} data type is that they are stored as 32-bit values instead of 64-bit values as \texttt{double}s are stored. This means that the program requires less memory and reduces the risk of triggering the \texttt{GC\_FOR\_ALLOC} due to the lack of free memory distributed to the process. Most modern devices has pipelined floating point operations to optimize calculations \cite{android:float}. In the case of the experiments performed in this thesis, \texttt{float} performed better than \texttt{double} for the device used in the tests.

The reason for this is that because \texttt{float}s are half the size of \texttt{double}s, they have a large chance of fitting in cache. Depending on the cache size, it could be beneficial to use the \texttt{float} data type. This is an important point to keep in mind. A reason why you would not use \texttt{float}s is when you need the precision of \texttt{double}s or when the performance difference is not an issue or non-existent.

The test results showed that there is a big difference when computing floats and doubles for in both Java and C++. It is also the case that some \texttt{float} tests in Java execute at half the execution time of the corresponding algorithm for \texttt{double} in C++. This shows the importance, on some architectures with different cache sizes, to choose the appropriate data structures as they can impact the performance greatly.
