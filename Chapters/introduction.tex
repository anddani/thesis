\textit{This thesis explores differences in performance between bytecode and natively compiled code. The Fast Fourier Transform algorithm is the main focus of this degree project. Experiments were carried out to investigate how and when it is necessary to implement the Fast Fourier Transform in Java or in C++ on Android.}

%%==================================================================
%% BACKGROUND
%%==================================================================
\section{Background}
Android is an operating system for smartphones and as of November 2016 it is the most used \cite{android:os:popularity}. One reason for this is because it was designed to be run on multiple different architectures \cite{android:os:devices}. Google states that they want to ensure that manufacturers and developers have an open platform to use and therefore releases Android as Open Source software \cite{android:os:opensource}. The Android kernel is based on the Linux kernel although with some alterations to support the hardware of mobile devices.

\gls{android} applications are mainly written in Java to ensure portability in form of architecture independence. By using a virtual machine to run a Java app, you can use the same bytecode on multiple platforms. To ensure efficiency on low resources devices, a virtual machine called Dalvik was developed. Applications (apps) on Android have been run on the Dalvik Virtual Machine (\gls{dvm}) up until Android version 5 \cite{android:dalvik} in November of 2014 \cite{android:dalvik:release}. Since then, Dalvik has been replaced by Android Runtime. Android Runtime, ART for short, differs from Dalvik in that it uses Ahead-Of-Time (\gls{aot}) compilation. This means that the bytecode is compiled during the installation of the app. Dalvik, however, exclusively uses a concept called Just-In-Time (\gls{jit}) compilation, meaning that code is compiled during runtime when needed. ART uses Dalvik bytecode to compile an application, allowing most apps that are aimed at DVM to work on devices running \gls{art}.

To allow developers to reuse libraries written in C or C++ or to write low level code, a tool called Native Development Kit (\gls{ndk}) was released. It was first released in June 2009 \cite{Lin2011} and has since gotten improvements such as new build tools, compiler versions and support for additional Application Binary Interfaces (\gls{abi}). ABIs are mechanisms that are used to allow binaries to communicate using specified rules. With the \gls{ndk}, developers can choose to write parts of an app in so called \emph{native code}. This is used when wanting to do compression, graphics and other performance heavy tasks.


%%==================================================================
%% PROBLEM
%%==================================================================
\section{Problem}
Nowadays, mobile phones are fast enough to handle heavy calculations on the devices themselves. To ensure that resources are spent in an efficient manner, this study has investigated how significant the performance boost is when compiling the Fast Fourier Transform (\gls{fft}) using the \gls{ndk} tools instead of using \gls{art}. Multiple implementations of FFTs were evaluated as well as the effects of the Java Native Interface (\gls{jni}), a framework for communicating between Java code and native static libraries. The following research question was formed on the basis of these topics:

\begin{center}
    \textit{Is there a significant performance difference between implementations of a Fast Fourier Transform (FFT) in native code, compiled by Clang, and Dalvik bytecode, compiled by Android Runtime, on Android?}
\end{center}

%%==================================================================
%% PURPOSE
%%==================================================================
\section{Purpose}
This thesis is a study that evaluates when and where there will be a gain in writing a part of an Android application in C++. One purpose of this study is to educate the reader about the cost, in performance and effort, of porting parts of an app to native code using the Native Development Kit (\gls{ndk}). Another is to explore the topic of performance differences between Android Runtime (\gls{art}) and native code compiled by Clang/\gls{llvm}. Because \gls{art} is relatively new (Nov 2014) \cite{android:dalvik:release}, this study would contribute with more information about to the performance of \gls{art} and how it performs compared to native code compiled by the \gls{ndk}. The results of the study can also be used to value the decision of implementing a given algorithm in native code instead of Java. It is valuable to know how efficient an implementation in native code is, depending on the size of the data.

The reason you would want to write a part of an application in native code is to potentially get better execution times on computational heavy tasks such as the Fast Fourier Transform (\gls{fft}). The \gls{fft} is an algorithm that computes the Discrete Fourier Transform (\gls{dft}) of a signal. It is primarily used to analyze the components of a signal. This algorithm is used in signal processing and has multiple purposes such as image compression (taking photos), voice recognition (Siri, Google Assistant), fingerprint scanning (unlocking device) to name a few. Another reason you would want to write native libraries is to reuse already written code in C or C++ and incorporate it into your project. This allows app functionality to become platform independent. Component code can then be shared with a computer program and an iOS app.

% MAYBE NOT RELEVANT?
% Some of the findings in this thesis can help decide which method of programming for Android that should be used for a given problem. For some problems, it is necessary to choose the appropriate programming method to ensure that an application is smooth and responsive. It is therefore important to know when and where it is necessary to optimize code. Further, when developing for Android there are multiple types of problems that occur and it is relevant to know which problems are worth solving in \gls{ndk} rather than the Software Development Kit (\gls{sdk}).

%%==================================================================
%% GOAL
%%==================================================================
\section{Goal}
The goal of this project was to examine the efficiency of ART and how it compares to natively written code using the \gls{ndk} in combination with the Java Native Interface (\gls{jni}). This report presents a study that investigates the relevance of using the \gls{ndk} to produce efficient code. Further, the cost to pass through the \gls{jni} is also a factor when analysing the code. A discussion about to what extent the efficiency of a program has an impact on the simplicity of the code is also present. For people who are interested to know about the impacts of implementing algorithms in C++ for Android, this study could be of some use.

%%==================================================================
%% METHOD
%%==================================================================
\section{Procedure}
%% KEYWORDS
%% NDK, Android, Benchmark*, Java, C, C++, Dalvik, Runtime, ART, efficien*, JNI,
%% FFT, Fast Fourier Transform, Fourier Transform, 

The method used to find the relevant literature and previous studies was to search through databases using boolean expressions. By specifying synonyms and required keywords, additional literature could be found. Figure~\ref{fig:db:search} contains an expression that was used to narrow down the search results to relevant articles.

\ifrelease
\begin{figure}[H]
    \centering
    \begin{align*}
        (\text{NDK OR JNI})               & \text{ AND } \\
        \text{Android}                    & \text{ AND } \\
        (\text{benchmark* OR efficien*})  & \text{ AND } \\
        (\text{Java OR C OR C++})         & \text{ AND } \\
        (\text{Dalvik OR Runtime OR ART}) &
    \end{align*}
    \caption{Expression used to filter out relevant articles}
    \label{fig:db:search}
\end{figure}
\fi

This is a strictly quantitative study, meaning numerical data and its statistical significance was the basis for the discussion. The execution time of the programs varied because of factors such as scheduling, CPU clock frequency scaling and other uncontrollable behaviour caused by the operating system. To get accurate measurements, a mean of a large numbers of runs were calculated for each program. Additionally, it was also necessary to calculate the standard error of each set of execution times. With the standard error we can determine if the difference in execution time between two programs are statistically significant or not.

Four different tests were carried out to gather enough data to be able to make reasonable statements about the results. The first one was to find out how significant the overhead of \gls{jni} is. This is important to know to be able to see exactly how large the cost of going between Java and native code is in relation to the actual work. The second test was a comparison between multiple well known libraries to find how much they differ in performance. In the third test, two comparable optimized implementations of \gls{fft}s were chosen, one recursive and one iterative in C++. These implementations were optimized using NEON, a vectorization library for the ARM architecture. In the fourth and final test, the \texttt{float} and \texttt{double} data types were compared.


%%==================================================================
%% DELIMITATIONS
%%==================================================================
\section{Delimitations}
This thesis does only cover a performance evaluation of the \gls{fft} algorithm and does not go into detail on other related algorithms. The decision of choosing the \gls{fft} was due to it being a common algorithm to use in digital signal analysis which is useful in many mobile applications. This thesis does not investigate the performance differences for \gls{fft} in parallel due to the complexity of the Linux kernel used on Android. This would require more knowledge outside the scope of this project and would result in this thesis being too broad. The number of optimization methods covered in this thesis were also delimited to the scope of this degree project.

\section{Limitations}
The tests were carried out on the same phone under the same circumstances to reduce the number of affecting factors. By developing a benchmark program that run the tests during a single session, it was possible to reduce the varying factors that could affect the results. Because you cannot control the Garbage Collector in Java, it is important to have this in mind when constructing tests and analyzing the data.

%%==================================================================
%% ETHICS AND SUSTAINABILITY
%%==================================================================
\section{Ethics and Sustainability}
An ethical aspect of this thesis is that because there could be people making decisions based on this report, it is important that the conclusions are presented together with its conditions so that there are no misunderstandings. Another important thing is that every detail of each test is explicitly stated so that every test can be recreated by someone else. Finally, it is necessary to be critical of the results and find how reasonable the results are.

Environmental sustainability is kept in mind in this investigation because there is an aspect of battery usage in different implementations of algorithms. The less number of instructions an algorithm require, the faster will the CPU lower its frequency, saving power. This will also have an influence on the user experience and can therefore have an impact on the society aspect of sustainability. If this study is used as a basis on a decision that have an economical impact, this thesis would fulfil the economical sustainability goal.

%%==================================================================
%% OUTLINE
%%==================================================================
\section{Outline}
\begin{itemize}
    \item \textit{\textbf{Chapter~\ref{ch:introduction} - Introduction --}} Introduces the reader to the project. This chapter describes why this investigation is beneficial in its field and for whom it is useful.
    \item \textit{\textbf{Chapter~\ref{ch:background} - Background --}} Provides the reader with the necessary information to understand the content of the investigation.
    \item \textit{\textbf{Chapter~\ref{ch:method} - Method --}} Discusses the hardware, software and methods that are the basis of the experiment. Here, the different methods of measurement presented and chosen.
    \item \textit{\textbf{Chapter~\ref{ch:results} - Results --}} The results of the experiments are presented here.
    \item \textit{\textbf{Chapter~\ref{ch:discussion} - Discussion --}} Discussion regarding the results as well as the chosen method.
    \item \textit{\textbf{Chapter~\ref{ch:conclusion} - Conclusion --}} Presents what the experiments showed and future work.
\end{itemize}
